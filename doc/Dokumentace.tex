\documentclass[12pt,oneside]{book} % use larger type; default would be 10pt

\usepackage[utf8]{inputenc} 
\usepackage[czech]{babel}
\usepackage[IL2]{fontenc}
\usepackage[a4paper]{geometry} 
\usepackage[pdftex]{graphicx} 
\usepackage[parfill]{parskip} 		% Activate to begin paragraphs with an empty line rather than an indent
\usepackage[explicit]{titlesec}
\usepackage{titletoc}
\usepackage{multirow}
\usepackage{bytefield}
\usepackage{rotating}
\usepackage{booktabs}  			% for much better looking tables
\usepackage{longtable}
\usepackage{lscape}
\usepackage{array} 	  		% for better arrays (eg matrices) in maths
\usepackage{paralist}  			% very flexible & customisable lists (eg. enumerate/itemize, etc.)
\usepackage{verbatim}  			% adds environment for commenting out blocks of text & for better verbatim
\usepackage{subfig}    			% make it possible to include more than one captioned figure/table in a single float
\usepackage[nottoc,notlof,notlot]{tocbibind} % Put the bibliography in the ToC
\usepackage[titles,subfigure]{tocloft}		 % Alter the style of the Table of Contents
\usepackage{color}
\usepackage[usenames,dvipsnames]{xcolor}
\usepackage{pdfmarginpar}
\usepackage{lastpage}
\usepackage{circuitikz}
\usepackage{tikz}
\usetikzlibrary{shapes,arrows,positioning,snakes,backgrounds,decorations.footprints,shadows,calc,chains} 
\usepackage{wrapfig}
\usepackage{listings}
\usepackage{tabularx}
\usepackage{enumitem}
\usepackage{anyfontsize}
\usepackage[pdftex, colorlinks=true, linkcolor=blue, urlcolor=blue]{hyperref} %should be last

\geometry{papersize={210mm,305mm},total={176mm,260mm}}

\usepackage{fancyhdr} % This should be set AFTER setting up the page geometry

\usepackage[local]{gitinfo2}

\pagestyle{fancyplain}     % options: empty , plain , fancy

\def\projectname{B5: Piano Tales Master}
\def\projectsubname{BROB - Základy robotiky\\[0.5cm]2019}
\def\projectdoc{Dokumentace projektu}

\definecolor{darkgray}{rgb}{0.4,0.4,0.9}
\definecolor{lightdarkgray}{rgb}{0.6,0.6,1.0}

\newcommand{\HRule}{\textcolor{darkgray}{\rule{\linewidth}{2mm}}}

\renewcommand{\headrulewidth}{1mm}
\renewcommand{\footrulewidth}{1mm}
\renewcommand{\plainheadrulewidth}{1mm}
\renewcommand{\plainfootrulewidth}{1mm}

\newcommand{\headrulecolor}{darkgray}
\newcommand{\footrulecolor}{darkgray}

\let\oldheadrule\headrule
\let\oldfootrule\footrule
\def\headrule{\textcolor{\headrulecolor}{\oldheadrule}}
\def\footrule{\textcolor{\footrulecolor}{\oldfootrule}}


\lhead{\projectdoc}
\chead{}
\rhead{\LARGE \projectname}
\lfoot{}
\cfoot{}
\rfoot{\thepage/\pageref{LastPage}}

\voffset 5mm
\setlength\headsep{8mm}

\newcommand{\up}[1]{\begin{sideways}\parbox{15mm}{#1}\end{sideways}}
\newcommand{\bx}[1]{\parbox{0.4\textwidth}{\centering #1}}

\tikzstyle{isipka} = [very thick,fill=none,rounded corners=2mm, color=red!100]
\tikzstyle{iodkaz} = [very thick,fill=none,rounded corners=3mm, color=red!100,line cap=round,align=center]
\tikzstyle{iprvek} = [rectangle, draw=red, rounded corners=3mm, line width=1mm]

\tikzstyle{postup} = [rectangle, draw, rounded corners=3mm, line width=1mm,minimum width=4cm,font=\bf]
\tikzstyle{labl} = [circle, draw, fill=white, rounded corners=3mm, line width=1mm,anchor=east,font=\bf]
\tikzstyle{edg} = [->, line width=1mm]

\tikzstyle{key} = [draw, fill=white, rectangle, rounded corners=2pt, inner sep=4pt, line width=0.5pt, drop shadow={shadow xshift=0.25ex,shadow yshift=-0.25ex,fill=black,opacity=0.75}, font=\scriptsize\sffamily, minimum height=1.5\baselineskip, minimum width=1.5\baselineskip]
\setcounter{secnumdepth}{-1}
\setcounter{tocdepth}{1}

\renewcommand\thepart{\arabic{part}}

\titleformat{\part}
  {\normalfont\normalsize}
  {}
  {20pt}
  {\begin{tikzpicture}[remember picture,overlay]%
	 \fill[lightdarkgray]  (current page.north west) rectangle ([yshift=-13cm]current page.north east);   
	 \node[fill=darkgray, text width=2\paperwidth, rounded corners=6cm, text depth=18cm, anchor=center, inner sep=0pt] at (current page.north east) (parttop) {\strut};
	 \node[anchor=south east, inner sep=0pt,	outer sep=0pt] at ([shift={(-2cm, +1cm)}]parttop.south)  (partnum) {\fontsize{10cm}{10}\selectfont\color{black}\thepart};
	 \node[anchor=north, inner sep=0pt]  at ([yshift=-2pt]partnum.south) (partname) {\large\scshape\bfseries\color{white} ČÁST};
 	 \node[anchor=north east, align=right, inner sep=0pt, outer ysep=2cm] at ([xshift=1cm]partnum.south) {\parbox{.7\textwidth}{\Huge\bfseries\raggedleft\projectname\\#1}};
 	 \node[anchor=north, align=left] at ([yshift=-12.5cm]current page.north) {\parbox{\textwidth}{\startcontents[part]\printcontents[part]{l}{0}{\setcounter{tocdepth}{0}}}};
   \end{tikzpicture}%    
  }

\titleformat{\chapter}
  {\normalfont}%
  {}%
  {20pt}%
  {\begin{tikzpicture}[remember picture,overlay]
    \fill[darkgray] (current page.north west) rectangle ([yshift=-3.5cm]current page.north east);
    \node[anchor=west, align=left, inner xsep=2cm] at ([yshift=-2cm]current page.north west) {\parbox{\textwidth}{\huge\sffamily\bfseries\scshape\color{white}#1}};%
   \end{tikzpicture}%
  }

\titlespacing*{\chapter}{0pt}{50pt}{-70pt}




\begin{document}
	
\begin{titlepage}
    \begin{center}

	~\\[0.5cm]
       \includegraphics[width=0.85\textwidth]{./img/uvodka.jpg}\\[0cm] 

	~\\[2cm]

    % Title
    \HRule \\[0.4cm]
    { \huge \bfseries \projectname}\\[0.4cm]
       \textsc{\LARGE \projectsubname}\\[0.4cm]
    \HRule \\[1cm]
    
    \textsc{\LARGE \projectdoc}\\[0.5cm]
    \texttt{\Large \gitFirstTagDescribe}\\[1.5cm]

    % Author and supervisor
    \begin{minipage}{0.5\textwidth}
      \begin{center} \large
        \includegraphics[width=0.85\textwidth]{./img/loga/vut.png}\\[1cm] 
		 \Large\bfseries UAMT FEKT VUT		
      \end{center}
    \end{minipage}%
    \begin{minipage}{0.5\textwidth}\raggedleft\Large\bfseries
 	Lukáš Zezula\par
        Dominik Řičánek\par
    \raggedright    
        Vedoucí projektu: \par
   \raggedleft     
        ing.Adam Ligocki
    \end{minipage}
    \vfill
   \end{center}
\end{titlepage}

\tableofcontents

%%%%%%%%%%%%%%%%%%%%%%%%%%%%%%%%%%%%%%%%%%%%%%%%%%%%%%%%%%%%%%%%%%%%%%%%%%%%%%%%%
%%%%%%%%%%%%%%%%%%%%%%%%%%%%%%%%%%%%%%%%%%%%%%%%%%%%%%%%%%%%%%%%%%%%%%%%%%%%%%%%%
%%%%%%%%%%%%%%%%%%%%%%%%%%%%%%%%%%%%%%%%%%%%%%%%%%%%%%%%%%%%%%%%%%%%%%%%%%%%%%%%%
%%%%%%%%%%%%%%%%%%%%%%%%%%%%%%%%%%%%%%%%%%%%%%%%%%%%%%%%%%%%%%%%%%%%%%%%%%%%%%%%%
\chapter{Analýza zadání}\label{analyza-zadani}
\section{Zadání}
\qquad Vytvořte robota, který bude hrát Piano Tales. Navrhněte stroj, který pomocí Vámi zvoleného HW (snímače a výpočetní jednotky a motorků) bude schopen hrát a porazit člověka ve hře Piano Tales. Referenčním výsledkem bude https://www.youtube.com/watch?v=fqOW84ZTL7k který se pokusíme porazit. Na závěr vznikne krátké propagační video na youtube obsahujcí logo VUT.
Projekt bude veden pomoci GITu. Dokumentace bude vytvořená v LaTeXu. 
\section{Úvod}\label{uvod}
\qquad Cílem projektu, jak je patrno ze zadání, bylo realizovat robota, který by hrál hru Piano Tales a dosahoval lepších výsledků než člověk a ideálně se přiblížil, nebo dokonce překonal referenční výsledek.

\qquad Hra Piano Tales funguje tak, že po obrazovce ve čtyřech lajnách jezdí obdelníky, jejichž stisknutím se zvýší skóre a rychlost s jakou se pohybují. Hra se tedy neustále ztěžuje. Jakmile nezachytíte již zmiňovaný obdelník nebo kliknete mimo jeho plochu, hra končí. Krom těchto obdelníků, které mají pevně definované rozměry, jezdí po obrazovce i delší obdelníkové úseky, které mají stálou šířku, ale jejich délka se různí \hyperref[pianotales]{(viz. Obr. 1)}. Tyto úseky se liší i barevně. Zatímco obdelníky jsou vždy striktně černé, tak barva úseků se mění od černé na začátku úseku po světlejší odstíny modré na konci. Přidržením těchto úseků od počátku do konce se skóre zvýší více než při pouhém stisknutí. 

\begin{center} \large\label{pianotales}
\includegraphics[width=0.35\textwidth]{./img/pianotales.png}\\[1cm] 	
\end{center}   
      
\begin{center}
Obr. 1 Ukázka hrací plochy hry Piano Tales
\end{center}
\section{Nástin řešení}\label{nastin}

%%%%%%%%%%%%%%%%%%%%%%%%%%%%%%%%%%%%%%%%%%%%%%%%%%%%%%%%%%%%%%%%%%%%%%%%%%%%%%%%%
%%%%%%%%%%%%%%%%%%%%%%%%%%%%%%%%%%%%%%%%%%%%%%%%%%%%%%%%%%%%%%%%%%%%%%%%%%%%%%%%%
%%%%%%%%%%%%%%%%%%%%%%%%%%%%%%%%%%%%%%%%%%%%%%%%%%%%%%%%%%%%%%%%%%%%%%%%%%%%%%%%%
%%%%%%%%%%%%%%%%%%%%%%%%%%%%%%%%%%%%%%%%%%%%%%%%%%%%%%%%%%%%%%%%%%%%%%%%%%%%%%%%%
\part{Návrh a realizace mechanické části}\label{mechanika}

\chapter{BLABLA}\label{BLABLA}
\section{2xBLABLA}\label{2xBLABLA}
%%%%%%%%%%%%%%%%%%%%%%%%%%%%%%%%%%%%%%%%%%%%%%%%%%%%%%%%%%%%%%%%%%%%%%%%%%%%%%%%%
%%%%%%%%%%%%%%%%%%%%%%%%%%%%%%%%%%%%%%%%%%%%%%%%%%%%%%%%%%%%%%%%%%%%%%%%%%%%%%%%%
%%%%%%%%%%%%%%%%%%%%%%%%%%%%%%%%%%%%%%%%%%%%%%%%%%%%%%%%%%%%%%%%%%%%%%%%%%%%%%%%%
%%%%%%%%%%%%%%%%%%%%%%%%%%%%%%%%%%%%%%%%%%%%%%%%%%%%%%%%%%%%%%%%%%%%%%%%%%%%%%%%%
\part{Návrh a realizace elektronické části}\label{elektro}

\chapter{BLABLA}\label{BLABLA}
\section{2xBLABLA}\label{2xBLABLA}

%%%%%%%%%%%%%%%%%%%%%%%%%%%%%%%%%%%%%%%%%%%%%%%%%%%%%%%%%%%%%%%%%%%%%%%%%%%%%%%%%
%%%%%%%%%%%%%%%%%%%%%%%%%%%%%%%%%%%%%%%%%%%%%%%%%%%%%%%%%%%%%%%%%%%%%%%%%%%%%%%%%
%%%%%%%%%%%%%%%%%%%%%%%%%%%%%%%%%%%%%%%%%%%%%%%%%%%%%%%%%%%%%%%%%%%%%%%%%%%%%%%%%
%%%%%%%%%%%%%%%%%%%%%%%%%%%%%%%%%%%%%%%%%%%%%%%%%%%%%%%%%%%%%%%%%%%%%%%%%%%%%%%%%
\part{Programová část}\label{program}

\chapter{BLABLA}\label{BLABLA}
\section{2xBLABLA}\label{2xBLABLA}

%%%%%%%%%%%%%%%%%%%%%%%%%%%%%%%%%%%%%%%%%%%%%%%%%%%%%%%%%%%%%%%%%%%%%%%%%%%%%%%%%
%%%%%%%%%%%%%%%%%%%%%%%%%%%%%%%%%%%%%%%%%%%%%%%%%%%%%%%%%%%%%%%%%%%%%%%%%%%%%%%%%
%%%%%%%%%%%%%%%%%%%%%%%%%%%%%%%%%%%%%%%%%%%%%%%%%%%%%%%%%%%%%%%%%%%%%%%%%%%%%%%%%
%%%%%%%%%%%%%%%%%%%%%%%%%%%%%%%%%%%%%%%%%%%%%%%%%%%%%%%%%%%%%%%%%%%%%%%%%%%%%%%%%
\part{Shrnutí a závěr}\label{shrnuti}

%%%%%%%%%%%%%%%%%%%%%%%%%%%%%%%%%%%%%%%%%%%%%%%%%%%%%%%%%%%%%%%%%%%%%%%%%%%%%%%%%
%%%%%%%%%%%%%%%%%%%%%%%%%%%%%%%%%%%%%%%%%%%%%%%%%%%%%%%%%%%%%%%%%%%%%%%%%%%%%%%%%
%%%%%%%%%%%%%%%%%%%%%%%%%%%%%%%%%%%%%%%%%%%%%%%%%%%%%%%%%%%%%%%%%%%%%%%%%%%%%%%%%
%%%%%%%%%%%%%%%%%%%%%%%%%%%%%%%%%%%%%%%%%%%%%%%%%%%%%%%%%%%%%%%%%%%%%%%%%%%%%%%%%
\part{Literatura}\label{literatura}

\end{document}
